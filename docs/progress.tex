%%
% 四次进度报告相关信息

% Author:   Souler Ou
% 修改者:    欧一锋
% Date:     3/30/2018
% Mail:     ou@souler.cc

% 第一次进度报告
\firstsummary{
	\begin{adjustwidth}{2em}{2em}
		在这一阶段,主要完成了如下几个方面的工作:
		\begin{enumerate}
			\item 将题目范围确定在曲面重构范围内,及多位尺度变换问题,根据点与点之间的距离信息,推出一组合理的坐标。
			\item 阅读教授提供的论文,梳理基本思路,了解问题的相关背景知识与现有解决方法,基本完成论文中第一章,即引言部分的编写。
			\item 查阅与多维尺度变换方法和低秩矩阵填充方法相关的参考文献资料,并进行总结。
		\end{enumerate}
	\end{adjustwidth}
}
% 第2次进度报告
\secondsummary{
	\begin{adjustwidth}{2em}{2em}
	    在这一阶段,主要完成了如下几个方面的工作:
	    \begin{enumerate}
	        \item 完成论文中经典多维尺度变换方法和SMACOF两种方法的理论部分编写,即对应论文正文中的第二章和第三章的内容。
	        \item 对经典多维尺度变换方法和SMACOF算法在MATLAB上进行实现,并根据教授所提供的数据集和对生成坐标显示结果的函数,初步判断方法的效果,为后面的数值实验做好准备。
	    \end{enumerate}
	\end{adjustwidth}
}
% 第3次进度报告
\thirdsummary{
	\begin{adjustwidth}{2em}{2em}
		\begin{enumerate}
		    \item 确定论文后半部分方法为先用低秩矩阵填充方法恢复距离信息矩阵,再使用经典多维尺度变换方法重构出坐标。并确定论文中要涉及的低维矩阵填充方法为SVT和OPTSPACE。
		    \item 完成论文正文部分中关于SVT和OPTSPACE的理论部分编写,即对应论文正文部分的第四章和第五章。
		    \item 对SVT和OPTSPACE两种方法在MATLAB上进行实现,并在教授提供的数据集上进行基础的数值实验。
		\end{enumerate}
	\end{adjustwidth}
}
% 第4次进度报告
\fourthsummary{
	\begin{adjustwidth}{2em}{2em}
		\begin{enumerate}
		    \item 对于前面关于OPTSPACE算法的方式进行补充,添加与Incremental OPTSPACE相关的内容,并完成相应论文中内容编写和MATLAB上程序的实现。
		    \item 在之前的数值实验结果的基础上,对各个算法进行调参操作,进行记录和总结。并完成数值实验部分,即论文正文部分中的第六章的初稿。
		    \item 在教授的指导下,在数值实验章节中适当添加分析比较的部分,并将论文题目确定为“多维尺度变换问题方法与分析”。
		    \item 补充第一章引言中内容和编写中英文摘要。
		\end{enumerate}
	\end{adjustwidth}
}
% 第1次老师评价
\firstcomment{
	\begin{adjustwidth}{2em}{2em}
	\begin{enumerate}
	    \item 建议先了解多维尺度问题的定义和简单经典模型。
	    \item 尝试理解在初始信息不完整时有什么多维尺度重构的方法。
	\end{enumerate}
	\end{adjustwidth}
}
% 第2次老师评价
\secondcomment{
	\begin{adjustwidth}{2em}{2em}
		\begin{enumerate}
		    \item SMACOF是经典的解决多维尺度问题的方法,可以将其理解并实现。 
		    \item 尝试使用MATLAB对实际的曲面信息进行读取和初步处理 \item 尝试对结果进行误差分析与图示。
		\end{enumerate}
	\end{adjustwidth}
}
% 第3次老师评价
\thirdcomment{
	\begin{adjustwidth}{2em}{2em}
		\begin{enumerate}
		    \item SVT是一种简单、高效的正则化方法,值得尝试。 
		    \item OPTSPACE是指导老师未曾直接实现的方法,但因为其前沿、效果好,可以进行尝试。
		    \item 尝试对于各种方法进行分析和比较。
		\end{enumerate}
	\end{adjustwidth}
}
% 第4次老师评价
\fourthcomment{
	\begin{adjustwidth}{2em}{2em}
	    \begin{enumerate}
	        \item 首先确保各个方法具有合理的参数,基本达到理论上的最优效果。 
	        \item 应当在获取结果后,加入必要的方法原理分析和效果比较,指明好的方法为什么好。
	        \item 对于结果图示应该确保规范、整齐、减少不必要的白边。
	    \end{enumerate}
	\end{adjustwidth}
}
% 老师总评价
\finalcomment{
	\begin{adjustwidth}{2em}{2em}
		
	\end{adjustwidth}
}